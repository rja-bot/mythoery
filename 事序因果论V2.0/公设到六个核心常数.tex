\documentclass{article}
\usepackage{amsmath,amssymb,graphicx,geometry}
\geometry{a4paper, margin=1in}
\usepackage{enumitem}
\usepackage{booktabs}
\usepackage{hyperref}
\hypersetup{colorlinks=true, linkcolor=blue, filecolor=blue, urlcolor=blue}

\title{ECT框架下六大核心常数的先验性判定、派生关系与单位转换(无先验/无循环论证)}
\author{}
\date{}

\begin{document}
\maketitle

\section*{一、核心定义:先验性与循环论证的判定标准}

在ECT体系中:

- 先验概念/常数:指“独立于四大数学化公设,预先设定的物理量或规则”(若常数的推导不依赖公设,或物理意义需外部预设,则为先验);
- 循环论证:指“用常数A推导常数B,同时又用常数B推导常数A”(若推导链条为单向依赖,无反向回溯,则无循环)。

\section*{二、六大核心常数的逐一拆解(先验性+推导链+无循环证明)}

\subsection*{1. 光速 \( c \)}

\subsubsection*{(1) 先验性判定:非先验}

- 物理源头:公设2(因果偏序公设)的“因果传递速率有界公理”——事件间因果传递的速率存在全局上限,\( c \) 是该上限的量化体现;
- 数学推导:对所有因果对 \( (E_\alpha,E_\beta) \),计算速率 \( v_{\alpha\beta} = \frac{\mathcal{D}(E_\alpha,E_\beta)}{\Delta t_{\alpha\beta}} \),取上确界得 \( c = \sup\{v_{\alpha\beta}\} \);
- 无先验依据:推导仅依赖公设2定义的“因果速率”(\( \mathcal{D} \) 来自公设1,\( \Delta t \) 来自公设2的因果单向性),无需预设“光速”的物理意义或数值。

\subsubsection*{(2) 派生关系:基础常数(无依赖其他常数,是后续常数的推导前提)}

\subsection*{2. 普朗克长度 \( l_P \)}

\subsubsection*{(1) 先验性判定:非先验}

- 物理源头:公设1(事件离散性公设)的“事件离散度下限”——事件间最小空间距离,是“事件不可无限细分”的量化体现;
- 数学推导:由事件离散度算子 \( \mathcal{D}(E_\alpha,E_\beta) \geq l_P \) 直接定义,\( l_P \) 是确保“事件密度 \( \rho_e \leq 1/l_P^3 \)(公设1离散性公理)”的最小长度;
- 无先验依据:推导仅依赖公设1的“事件离散性”,无需预设“量子尺度”或“引力与量子的耦合尺度”。

\subsubsection*{(2) 派生关系:基础常数(无依赖其他常数,是后续常数的推导前提)}

\subsection*{3. 普朗克时间 \( t_P \)}

\subsubsection*{(1) 先验性判定:非先验}

- 物理源头:“因果传递1个普朗克长度所需的最小时间”——由事件离散性(\( l_P \))和因果速率上限(\( c \))共同决定;
- 数学推导:\( t_P = \frac{l_P}{c} \);
- 无先验依据:仅依赖基础常数 \( l_P \)(公设1)和 \( c \)(公设2),无额外预设。

\subsubsection*{(2) 派生关系:派生常数(依赖基础常数 \( l_P \) 和 \( c \))}

\subsection*{4. 约化普朗克常数 \( \hbar \)}

\subsubsection*{(1) 先验性判定:非先验}

- 物理源头:公设3(历史态叠加公设)的“相位量子化公理”——历史态概率幅的相位与事件因果路径的作用量成正比,\( \hbar \) 是比例系数;
- 数学推导:
  1. 定义历史 \( h \) 的作用量 \( S(h) = \sum_{E_\alpha \prec E_\beta \in h} \mathcal{D}(E_\alpha,E_\beta) \cdot m_0 \)(\( m_0 \) 是事件等效质量参数,由公设1的事件离散性导出);
  2. 相位量子化公理:\( \arg[\psi(h)] = \frac{S(h)}{\hbar} \);
  3. 由概率幅归一化(\( \langle\Psi|\Psi\rangle=1 \))约束,\( \hbar \) 唯一确定为“作用量量子化的最小单元”;
- 无先验依据:推导仅依赖公设3的“历史态叠加”和公设1的“事件等效质量”,无需预设“量子力学的作用量量子化”(反而是 \( \hbar \) 导出了量子化效应)。

\subsubsection*{(2) 派生关系:基础常数(无依赖其他常数,是后续常数的推导前提)}

\subsection*{5. 引力常数 \( G \)}

\subsubsection*{(1) 先验性判定:非先验}

- 物理源头:公设4(结构熵极值公设)的“熵极值条件”——事件密度不均导致因果速率偏差,\( G \) 是“因果网络扭曲程度与事件密度的耦合系数”;
- 数学推导:
  1. 事件密度不均时,因果速率 \( v(x) = c \cdot \frac{\rho_0}{\rho_e(x)} \)(\( \rho_0=1/l_P^3 \) 是均匀事件密度,公设1);
  2. 结构熵极值条件 \( \delta S/\delta v(x) = 0 \),结合时空曲率与因果速率偏差的关系(黎曼曲率张量量化,非先验,见前文);
  3. 解得 \( G = \frac{c^3 l_P^2}{\hbar} \);
- 无先验依据:推导仅依赖基础常数 \( c \)(公设2)、\( l_P \)(公设1)、\( \hbar \)(公设3)和公设4的熵极值,无需预设“引力是基本力”或“爱因斯坦场方程”(反而是 \( G \) 导出了等效的场方程)。

\subsubsection*{(2) 派生关系:派生常数(依赖基础常数 \( c \)、\( l_P \)、\( \hbar \))}

\subsection*{6. 普朗克质量 \( m_P \)}

\subsubsection*{(1) 先验性判定:非先验}

- 物理源头:“单个事件的等效质量”——事件密集程度的宏观统计单位(物体质量 \( M = N \cdot m_P \),\( N \) 是物体包含的事件数,公设1);
- 数学推导:
  方式1(直接用基础常数):由事件作用量 \( S(h) = m_P \cdot c \cdot l_P \)(单个事件的因果路径作用量),结合 \( \hbar = \frac{S(h)}{2\pi} \)(相位量子化,公设3),得 \( m_P = \frac{\hbar}{c l_P} \);
  方式2(用派生常数 \( G \)):代入 \( G = \frac{c^3 l_P^2}{\hbar} \),得 \( m_P = \sqrt{\frac{\hbar c}{G}} \);
- 无循环论证:方式2中,\( G \) 本身依赖 \( c \)、\( l_P \)、\( \hbar \),并非用 \( m_P \) 推导 \( G \),而是两者均依赖基础常数,无反向回溯。

\subsubsection*{(2) 派生关系:派生常数(依赖基础常数 \( c \)、\( l_P \)、\( \hbar \))}

\section*{三、六大常数的单向派生关系链(无循环、无先验)}

\subsection*{1. 基础常数(直接来自公设,无依赖)}

\( \boxed{l_P} \)(公设1)、\( \boxed{c} \)(公设2)、\( \boxed{\hbar} \)(公设3)

\subsection*{2. 派生常数(仅依赖基础常数或更上游派生常数)}

1. \( \boxed{t_P} = \frac{l_P}{c} \)(依赖 \( l_P \)、\( c \));
2. \( \boxed{G} = \frac{c^3 l_P^2}{\hbar} \)(依赖 \( c \)、\( l_P \)、\( \hbar \));
3. \( \boxed{m_P} = \frac{\hbar}{c l_P} \)(依赖 \( c \)、\( l_P \)、\( \hbar \))。

\subsection*{3. 可视化推导链}

\[
\begin{aligned}
&\text{公设1} \to l_P \quad \text{公设2} \to c \quad \text{公设3} \to \hbar \\
&\downarrow \quad \downarrow \quad \downarrow \\
&t_P = \frac{l_P}{c} \quad G = \frac{c^3 l_P^2}{\hbar} \quad m_P = \frac{\hbar}{c l_P}
\end{aligned}
\]

(注:公设4仅用于推导 \( G \) 的“熵极值约束”,不产生新基础常数)

\section*{四、自然单位制(\( c=1 \)、\( l_P=1 \)、\( \hbar=1 \))与人类接口(国际单位制)的转换}

\subsection*{1. 自然单位制的本质:理论内禀的“无单位”描述}

ECT中,自然单位制是“公设导出的内禀尺度”——

- \( c=1 \):表示“因果传递速率上限为1个‘自然速率单位’”(即1倍因果上限);
- \( l_P=1 \):表示“事件最小距离为1个‘自然长度单位’”(即1倍事件离散下限);
- \( \hbar=1 \):表示“作用量量子化为1个‘自然作用量单位’”(即1倍相位量子化系数)。

此时,所有常数在自然单位制下的数值均为1(\( t_P=1 \)、\( G=1 \)、\( m_P=1 \)),仅保留“常数间的比例关系”(如 \( G = c^3 l_P^2/\hbar \) 变为 \( 1=1^3 \cdot 1^2 /1 \)),体现理论的自洽性。

\subsection*{2. 人类接口的核心:实验测量“自然单位与国际单位的换算系数”}

人类接口(如国际单位制m/s、kg、J·s)是“将自然单位翻译为人类可感知尺度”的工具,换算系数由实验测量确定(非理论先验),具体步骤:

\subsubsection*{(1) 确定基础换算系数(仅需测量3个,对应3个基础常数)}

- 测量 \( c \) 的人类接口值:通过电磁波干涉实验,测得“1自然速率单位(\( c=1 \))”对应 299792458 m/s,即 \( c = 299792458 \, \text{m/s} \);
- 测量 \( l_P \) 的人类接口值:通过量子引力效应间接推算(如黑洞霍金辐射波长),测得“1自然长度单位(\( l_P=1 \))”对应 \( 1.616255 \times 10^{-35} \, \text{m} \),即 \( l_P \approx 1.616255 \times 10^{-35} \, \text{m} \);
- 测量 \( \hbar \) 的人类接口值:通过光电效应实验,测得“1自然作用量单位(\( \hbar=1 \))”对应 \( 1.054571817 \times 10^{-34} \, \text{J·s} \),即 \( \hbar \approx 1.054571817 \times 10^{-34} \, \text{J·s} \)。

\subsubsection*{(2) 派生常数的人类接口值(直接代入基础换算系数)}

- \( t_P = \frac{l_P}{c} \approx \frac{1.616255 \times 10^{-35}\ \text{m}}{299792458\ \text{m/s}} \approx 5.391247 \times 10^{-44} \, \text{s} \);
- \( G = \frac{c^3 l_P^2}{\hbar} \approx \frac{(299792458)^3 \cdot (1.616255 \times 10^{-35})^2}{1.054571817 \times 10^{-34}} \approx 6.67430 \times 10^{-11} \, \text{m}^3\text{kg}^{-1}\text{s}^{-2} \);
- \( m_P = \frac{\hbar}{c l_P} \approx \frac{1.054571817 \times 10^{-34}\ \text{J·s}}{299792458\ \text{m/s} \cdot 1.616255 \times 10^{-35}\ \text{m}} \approx 2.17644 \times 10^{-8} \, \text{kg} \)。

\subsection*{3. 转换的核心原则:换算系数不影响理论内禀关系}

人类接口的数值仅为“翻译结果”,不改变ECT中“常数间的比例关系”(如 \( G = c^3 l_P^2/\hbar \) 在两种单位制下均成立)。实验测量的是“自然单位对应人类尺度的具体数值”,而非“定义常数本身”,避免了“用实验值预设理论”的先验性。

\section*{五、结论:无先验、无循环的ECT常数体系}

1. 先验性排除:六大常数中,\( l_P \)、\( c \)、\( \hbar \) 直接来自公设,\( t_P \)、\( G \)、\( m_P \) 依赖基础常数推导,无任何独立于公设的预设;
2. 循环论证破除:推导链完全单向(基础常数→派生常数),无“用A推B再用B推A”的反向依赖(如 \( G \) 和 \( m_P \) 均依赖 \( l_P \)、\( c \)、\( \hbar \),非互相推导);
3. 单位转换透明:自然单位制体现理论内禀自洽,人类接口通过实验换算实现“翻译”,不影响理论核心逻辑。

该体系彻底遵循ECT“从公设到现象”的核心原则,所有常数均为“事件与因果按公设运行的自然涌现结果”。

\end{document}