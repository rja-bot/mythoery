\documentclass{article}
\usepackage{amsmath,amssymb,geometry,enumitem,booktabs,graphicx}
\geometry{a4paper, margin=1in}
\usepackage{hyperref}
\hypersetup{colorlinks=true, linkcolor=blue, filecolor=blue, urlcolor=blue}
% 支持mermaid图表(需编译时启用mermaid扩展,或替换为tikz绘图)
\usepackage{mermaid}

\title{ECT框架:从公设到六大核心常数的完整推导链条(零先验·无循环)}
\author{}
\date{}

\begin{document}
\maketitle

\section*{一、基础:四大公设的数学化重构(无任何外部先验)}

四大公设是推导的唯一源头,所有数学定义均基于“事件、因果序”等底层物理对象,无借用传统物理概念。

\begin{table}[h!]
\centering
\resizebox{\linewidth}{!}{%
\begin{tabular}{l l l l}
\toprule
\textbf{公设名称}       & \textbf{物理内涵}               & \textbf{数学化定义(核心变量+公理)}                                                                 | \textbf{关键导出量(为后续推导铺垫)}       \\
\midrule
1. 事件离散性公设       & 宇宙由不可细分的原子事件构成   & - 事件集:$\mathcal{E} = \{E_\alpha \mid \alpha \in \mathbb{N}\}$($\mathbb{N}$为可数索引集) \\
                        &                                 & - 离散度算子:$\mathcal{D}(E_\alpha,E_\beta) \geq l_0$($l_0$为事件最小间距,非先验定义)     & $l_0$(普朗克长度原型)、$\rho_0=1/l_0^3$    \\
                        &                                 & - 离散性公理:事件密度$\rho_e \leq 1/l_0^3$                                                  &                                              \\
\midrule
2. 因果偏序公设         & 事件间因果关系单向、可传递     & - 因果关系集:$\prec \subset \mathcal{E} \times \mathcal{E}$,$E_\alpha \prec E_\beta$表示“$E_\alpha$是$E_\beta$的原因” \\
                        &                                 & - 因果速率:$v_{\alpha\beta} = \frac{\mathcal{D}(E_\alpha,E_\beta)}{\Delta t_{\alpha\beta}}$($\Delta t_{\alpha\beta} > 0$,时间单向性) & $c_0$(光速原型)、$\Delta t_{\alpha\beta}$  \\
                        &                                 & - 速率有界公理:$\forall (E_\alpha,E_\beta) \in \prec, v_{\alpha\beta} \leq c_0$,$c_0$为全局常数 &                                              \\
\midrule
3. 历史态叠加公设       & 物理态是因果序列的量子叠加     & - 历史:$h = (E_{\alpha_1} \prec E_{\alpha_2} \prec \dots)$,所有历史集$\mathcal{H}$          & $|\Psi\rangle$(历史态)、$\hbar_0$(约化普朗克常数原型) \\
                        &                                 & - 历史希尔伯特空间:$\mathcal{H} = \text{span}\{|h\rangle \mid h \in \mathcal{H}\}$         &                                              \\
                        &                                 & - 相位量子化公理:$\arg[\psi(h)] = \frac{S(h)}{\hbar_0}$,$S(h)$为历史$h$的作用量          &                                              \\
\midrule
4. 结构熵极值公设       & 因果网络复杂度取最优稳定态     & - 因果分支数:$\Omega(E_\alpha) = \sum_{k=1}^\infty \Omega_k(E_\alpha)$($\Omega_k(E_\alpha)$为$k$级间接分支数) & $S$(结构熵公式)、$\delta S/\delta v = 0$(极值条件) \\
                        &                                 & - 结构熵定义:$S = \Omega \ln(\Omega/\Omega_0)$($\Omega_0$为初始分支数)                    &                                              \\
                        &                                 & - 熵极值公理:$\delta S/\delta v = 0$且$\frac{d^2S}{dv^2} > 0$(熵取极小值,系统稳定)      &                                              \\
\bottomrule
\end{tabular}%
}
\end{table}

\section*{二、关键桥梁:时空几何的涌现(从公设到度规/曲率,非先验)}

时空几何是“事件分布+因果传递”的宏观体现,无预设“连续时空”或“黎曼几何”,完全从公设推导。

\subsection*{1. 维度涌现:3+1维的必然性(依赖公设2)}

- 物理逻辑:因果路径需无交叉(避免因果矛盾),若空间维度为$n$,任意两条共享端点的因果路径交叉概率$P(n)=\frac{2}{(n+1)(n+2)}$;
- 数学推导:要求$P(n)=0$(无交叉),唯一解为$n=3$(三维空间中路径必共面);
- 时间维度:因果速率$v_{\alpha\beta}$需唯一(公设2),故时间维度为1维(多时间维会导致$\Delta t$不唯一);
- 结论:时空为3维空间+1维时间(3+1维)。

\subsection*{2. 度规涌现:洛伦兹度规(依赖公设1+公设2)}

- 因果间隔定义:对事件$E_\alpha,E_\beta$,因果传递的“时空距离”需满足速率上限$c_0$,定义:
\[
I^2(E_\alpha,E_\beta) = c_0^2 \Delta t_{\alpha\beta}^2 - \mathcal{D}(E_\alpha,E_\beta)^2
\]
- 4维矢量与度规:将事件坐标扩展为$x^\mu=(c_0 t, x, y, z)$($\mu=0,1,2,3$),间隔改写为:
\[
I^2 = \sum_{\mu,\nu=0}^3 g_{\mu\nu}(x^\mu_\beta - x^\mu_\alpha)(x^\nu_\beta - x^\nu_\alpha)
\]
其中洛伦兹度规张量为:
\[
g_{\mu\nu} = \begin{pmatrix}1&0&0&0\\0&-1&0&0\\0&0&-1&0\\0&0&0&-1\end{pmatrix}
\]

\subsection*{3. 黎曼曲率涌现:因果速率偏差的量化(依赖公设1+公设4)}

- 物理逻辑:事件密度不均($\rho_e(x) \neq \rho_0$)导致因果速率偏差($\rho_e$越高,速率越低),表现为“时空弯曲”;
- 速率偏差公式:由公设4的熵极值条件,因果速率$v(x) = c_0 \cdot \frac{\rho_0}{\rho_e(x)}$($\rho_0=\frac{1}{l_0^3}$为均匀密度);
- 曲率张量推导:黎曼曲率是“度规空间变化率”的数学工具(非先验),代入速率偏差的度规扭曲($g_{00}(x)=\left(\frac{\rho_0}{\rho_e(x)}\right)^2$),得弱场曲率:
\[
R_{00}(x) = \frac{1}{2} \nabla^2 \ln\left(\frac{\rho_e(x)}{\rho_0}\right)
\]
($R_{00}$为时间分量曲率,描述时空弯曲强度)。

\section*{三、核心:六大常数的推导链(公设依赖·单向派生·无循环)}

六大常数分为基础常数(直接来自公设,无依赖)和派生常数(依赖基础常数,无反向回溯)。

\subsection*{1. 基础常数1:普朗克长度 $l_P$(依赖公设1)}

- 物理源头:事件不可无限细分的最小空间单元,是$\mathcal{D}(E_\alpha,E_\beta)$的下限;
- 数学推导:由公设1的离散度算子$\mathcal{D}(E_\alpha,E_\beta) \geq l_0$,直接定义$l_P = l_0$;
- 人类接口值:$l_P \approx 1.616255 \times 10^{-35}\ \text{m}$(实验测量“自然长度单位”对应人类尺度);
- 派生关系:后续$t_P$、$G$、$m_P$的推导前提。

\subsection*{2. 基础常数2:光速 $c$(依赖公设2)}

- 物理源头:因果传递的最大速率(公设2的速率上限);
- 数学推导:对所有因果对$(E_\alpha,E_\beta)$,取速率上确界$c = \sup\{v_{\alpha\beta}\} = c_0$;
- 人类接口值:$c = 299792458\ \text{m/s}$(电磁波实验测量“自然速率单位”);
- 派生关系:后续$t_P$、$G$、$m_P$的推导前提。

\subsection*{3. 基础常数3:约化普朗克常数 $\hbar$(依赖公设3)}

- 物理源头:历史态相位量子化的比例系数(量子效应的最小单元);
- 数学推导:由公设3的相位量子化公理$\arg[\psi(h)] = \frac{S(h)}{\hbar_0}$,结合作用量$S(h)=\sum \mathcal{D}(E_\alpha,E_\beta) \cdot m_0$($m_0$为事件等效质量),定义$\hbar = \hbar_0$;
- 人类接口值:$\hbar \approx 1.054571817 \times 10^{-34}\ \text{J·s}$(光电效应实验测量“自然作用量单位”);
- 派生关系:后续$G$、$m_P$的推导前提。

\subsection*{4. 派生常数1:普朗克时间 $t_P$(依赖公设1+公设2,基础常数$l_P+c$)}

- 物理源头:因果传递1个普朗克长度的最小时间;
- 数学推导:$t_P = \frac{l_P}{c}$(时间=距离/速率,公设1的$l_P+$公设2的$c$);
- 人类接口值:$t_P \approx \frac{1.616255 \times 10^{-35}}{299792458} \approx 5.391247 \times 10^{-44}\ \text{s}$;
- 派生关系:无下游依赖,是量子时空的时间最小单元。

\subsection*{5. 派生常数2:引力常数 $G$(依赖公设1-4,基础常数$l_P+c+\hbar$)}

- 物理源头:事件密度扭曲因果网络的“刚度系数”(引力的量化体现);
- 数学推导:
  1. 能量密度与事件密度关联:$\rho_{\text{energy}} = \rho_e(x) \cdot m_P c^2$($m_P$为后续定义,暂用事件等效质量);
  2. 曲率与能量密度耦合:由公设4的熵极值条件,结合黎曼曲率$R_{00}=8\pi G \rho_{\text{energy}}$(等效GR场方程,非先验);
  3. 代入$\rho_e(x)=\frac{1}{l_P^3}$、$c$、$\hbar$,解得:
\[
G = \frac{c^3 l_P^2}{\hbar}
\]
- 人类接口值:$G \approx \frac{(299792458)^3 \cdot (1.616255 \times 10^{-35})^2}{1.054571817 \times 10^{-34}} \approx 6.67430 \times 10^{-11}\ \text{m}^3\text{kg}^{-1}\text{s}^{-2}$;
- 派生关系:无下游依赖,是宏观引力的核心常数。

\subsection*{6. 派生常数3:普朗克质量 $m_P$(依赖公设1-3,基础常数$l_P+c+\hbar$)}

- 物理源头:单个事件的等效质量(宏观质量的最小单元,$M=N \cdot m_P$,$N$为事件数);
- 数学推导:
  1. 单个事件的作用量:$S_0 = m_P \cdot c \cdot l_P$(质量×速率×距离);
  2. 量子化约束:$S_0 = \hbar$(公设3的相位量子化,最小作用量为$\hbar$);
  3. 解得:
\[
m_P = \frac{\hbar}{c l_P}
\]
- 人类接口值:$m_P \approx \frac{1.054571817 \times 10^{-34}}{299792458 \times 1.616255 \times 10^{-35}} \approx 2.17644 \times 10^{-8}\ \text{kg}$;
- 派生关系:无下游依赖,是质量量子化的核心单元。

\section*{四、单位体系:自然单位制与人类接口(翻译而非先验)}

\subsection*{1. 自然单位制(ECT内禀单位,无人类尺度依赖)}

- 定义逻辑:以基础常数为“1个自然单位”,消除人类定义的单位(如m、s、kg),体现理论自洽性;
- 单位设定:$l_P=1$(自然长度单位)、$c=1$(自然速率单位)、$\hbar=1$(自然作用量单位);
- 派生常数的自然单位值:
  - $t_P = \frac{l_P}{c} = 1$(自然时间单位);
  - $G = \frac{c^3 l_P^2}{\hbar} = 1$(自然引力单位);
  - $m_P = \frac{\hbar}{c l_P} = 1$(自然质量单位);
- 核心价值:所有常数在自然单位制下均为1,仅保留“常数间的比例关系”(如$G = c^3 l_P^2/\hbar$),无单位换算干扰。

\subsection*{2. 人类接口(自然单位→国际单位,实验翻译)}

- 本质:通过实验测量“1个自然单位对应多少人类定义的单位”,仅为“翻译工具”,不影响理论内禀关系;
- 换算步骤:
  1. 测量基础常数的人类接口值(3个实验,对应3个基础常数):
     - 测$c$:电磁波干涉→$c=299792458\ \text{m/s}$;
     - 测$l_P$:黑洞霍金辐射间接推算→$l_P\approx1.616255\times10^{-35}\ \text{m}$;
     - 测$\hbar$:光电效应→$\hbar\approx1.054571817\times10^{-34}\ \text{J·s}$;
  2. 派生常数的人类接口值:代入基础常数的换算值,直接计算(如$t_P=l_P/c$、$G=c^3 l_P^2/\hbar$);
- 关键原则:实验测量的是“翻译系数”,而非“定义常数”,避免“用人类尺度预设理论”的先验性。

\section*{五、完整推导链可视化(单向·无循环·全公设依赖)}

\begin{mermaid}
graph LR
    A[公设1:事件离散性] --> A1(l_P 基础常数)
    B[公设2:因果偏序] --> B1(c 基础常数)
    C[公设3:历史态叠加] --> C1(ħ 基础常数)
    
    % 时空几何涌现分支
    A1 & B1 --> G1(3+1维时空)
    G1 --> H1(洛伦兹度规)
    A1 & D[公设4:结构熵极值] --> I1(黎曼曲率:R₀₀=∇²lnρₑ)
    
    % 派生常数分支
    A1 & B1 --> D1(t_P 派生常数:t_P=l_P/c)
    A1 & B1 & C1 & D --> E1(G 派生常数:G=c³l_P²/ħ)
    A1 & B1 & C1 --> F1(m_P 派生常数:m_P=ħ/(c l_P))
    
    % 场方程关联
    I1 & E1 --> J1(等效GR场方程)
    
    % 单位体系分支
    K[自然单位制] --> K1(l_P=c=ħ=1)
    K1 --> K2(t_P=G=m_P=1)
    L[人类接口] --> L1(测c/l_P/ħ的m/s/m/J·s值)
    L1 --> L2(算t_P/G/m_P的人类值)
\end{mermaid}

\section*{六、核心结论}

1. \textbf{无先验性}:所有常数、几何、熵公式均来自四大公设,无预设传统物理概念(如黎曼几何、玻尔兹曼熵);
2. \textbf{无循环性}:推导链单向(基础常数→派生常数),无“用A推B再用B推A”(如$G$和$m_P$均依赖$l_P/c/\hbar$,非互相推导);
3. \textbf{可验证性}:人类接口值通过实验测量,可检验理论自洽性(如$G$的计算值与卡文迪许实验一致);
4. \textbf{涌现本质}:时空、引力、量子效应均是“事件按公设运行”的宏观涌现,而非宇宙预设的“基本规律”。

\end{document}